\section{Introduction}
Unmanned Surface Vehicles (USVs) play a crucial role in commercial shipping\cite{USVshipping}, environmental 
and climate monitoring\cite{USVenvmonitor}, seabed mapping\cite{USVseabedmap}, surveillance and infrastructure 
inspection\cite{USVbridgeinspection}. USVs and AUVS (autonomous underwater vehicles) share similar benefits, 
particularly in reducing the need for personnel to operate in hazardous environments, thereby enhancing safety 
and security and lowering operational costs for maritime tasks. However, AUVs are limited in range and flexibility 
due to their reliance on the mother ship for positioning support\cite{AUV}. In contrast, USVs can operate over 
a wider range with greater precision, offering increased autonomy and adaptability in complex environments 
\cite{USVoverview}. 

As the advantages of USVs become increasingly evident, navigation has emerged as a critical area of 
research. Due to the expansive spatial scale of the marine environment, LiDAR Point Cloud systems commonly 
used in UGVs (unmanned ground vehicles) are not suitable for USVs \cite{lidarpointcloud}. Therefore, cameras 
are employed as sensing modules for USVs due to their lightweight, low-power, and high-information-density 
characteristics \cite{MODS}. Additionally, a perception module is needed to extract useful information from 
high-information-density images through computer vision \cite{perceptionmodule}. Semantic segmentation, which 
provides pixel-level semantic information, is particularly valuable as it helps intelligent systems understand 
spatial positions and make critical decisions \cite{SSsurvey}. This approach, which has been favored in autonomous 
driving\cite{autodrive1}, \cite{autodrive2}, is applicable to USVs with deep learning techniques.

However, using semantic segmentation on USVs presents two main challenges. First, the accuracy of semantic 
segmentation declines when USVs encounter novel scenarios. Most deep learning models for semantic segmentation 
produce point estimates as outputs without providing confidence levels \cite{evaluateBDL}, which can lead to 
errors in novel environments. For example, the autonomous driving system mistakenly identified a plastic bag as a 
rock, resulting in an unnecessary emergency brake \cite{autodrive-mistake}.
% two African Americans were mistakenly identified as gorillas by an image classification system \cite{USAtoday}. 
If models could provide lower confidence for uncertain predictions, 
such issues could be mitigated. Secondly, there is a scarcity of per-pixel semantically labelled marine environment 
databases. Due to the complexity of marine environments, collecting marine datasets is challenging, and the 
available marine datasets are less than those for land environments \cite{CamVid}, \cite{Cityscapes}, 
\cite{landsemanticdataset}. The requirement for per-pixel semantic labelling further increases the cost of marine 
datasets. 

Probabilistic outputs can be derived using the softmax function as the activation function at the output layer, 
or through more complex methods like particle filters to assist in modelling uncertainty in computer vision as a 
measure of confidence \cite{particlefilter}. However, these methods are not based on deep learning and cannot 
achieve state-of-the-art performance \cite{resnet}. Bayesian deep learning approaches have proven effective 
for capturing uncertainty in semantic segmentation, offering a more refined way to model uncertainty 
\cite{whatuncertaintydoweneed}. Additionally, Bayesian deep learning can also address dataset scarcity by 
incorporating epistemic uncertainty to reflect confidence in novel scenarios and using prior distributions 
for regularization, thus preventing overfitting \cite{overfitting}. 

In selecting the deep learning architecture, the VGG16-based SegNet approach was chosen due to the availability 
of existing Bayesian SegNet methodologies that can be leveraged \cite{VGG,SegNet}. The VGG architecture, being 
simpler compared to alternatives like ResNet, is easier to implement and serves well as a foundational model for 
research \cite{vggcompare}. 

The aim of this research is to enhance the visual detection capabilities of USVs by utilizing Bayesian 
SegNet for semantic segmentation in marine environments. This demonstrates that Bayesian SegNet 
can provide robust uncertainty estimates, enhancing reliability in complex and dynamic marine environments. 
Additionally, this approach addresses the challenges posed by dataset scarcity, improving the performance of 
USV perception systems in novel scenarios.

The remainder of this thesis is structured as follows. Section \ref{section:LR} reviews and discusses the most 
relevant prior research in the field. A comprehensive overview of the methodology employed for semantic segmentation
is provided in Section \ref{sectino:Methodology}. Evaluation metrics are presented in Section \ref{section:results}, 
along with a multidimensional performance comparison. Section \ref{section:discussion} analyses the evalaution 
performance and proposes future research directions, while Section \ref{section:conclusion} offers a comprehensive 
conclusion.