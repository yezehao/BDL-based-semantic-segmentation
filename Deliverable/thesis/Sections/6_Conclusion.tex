\section{Conclusion}
\label{section:conclusion}
This study aimed to enhance semantic segmentation for USVs by utilizing Bayesian SegNet to provide uncertainty 
estimation and improve robustness in novel environments, thereby addressing the challenge of limited maritime 
environment semantic segmentation datasets. The results indicate that Bayesian SegNet, by reducing noise in input 
data, improves Precision ($\mathbf{Pr}$) by 1.3\% and increases the F1 score ($\mathbf{F1}$) by 6.5\% compared to the 
non-Bayesian baseline. Additionally, Bayesian SegNet significantly outperforms SegNet in uncertainty estimation, as 
demonstrated through entropy-based analysis. 

Moreover, the Bayesian model shows superior generalization capabilities, with a notable advantage in evaluating 
the OASIs dataset from the USV perspective. The F1 score achieved by Bayesian SegNet surpasses that of SegNet by 
39.77\%, highlighting the strong inference abilities of Bayesian Approach in conditions with limited training data. 
These evaluations further demonstrate that Bayesian SegNet can effectively enhance USV environmental perception 
based on existing trained models in new environments. 

The characteristics of Bayesian SegNet provide USVs with more precise and reliable environmental perception, 
enhancing their autonomous navigation and risk avoidance capabilities in maritime environment. Given the 
ever-changing maritime environment, the coverage of available datasets is inherently limited. By leveraging the 
inference capabilities of the Bayesian approach, this model can effectively perceive a wider range of scenarios 
using limited datasets. As part of the perception module, this model is crucial for the autonomous navigation 
systems of USVs.

Beyond maritime applications, the methods, and findings of this study have the potential to impact other USV-related 
fields where strong environmental perception is essential. For instance, automated navigation in inland waterways 
could benefit from the improved uncertainty estimation and generalization capabilities provided by Bayesian SegNet, 
enabling safer and more reliable navigation in complex urban channels. Additionally, in fields with a significant 
scarcity of datasets, such as underwater robotics navigation, decision-making under uncertainty is crucial. 
Integrating similar Bayesian approaches into their perception systems could lead to substantial advancements.

However, this study also acknowledges certain limitations. Firstly, the model's parameter tuning, particularly 
the selection of various thresholds, requires further refinement. Additionally, other datasets should be used to 
further verify the generalization capabilities. Future work should focus on testing the model with different datasets, 
including MODD2 and SMD, and enhancing the uncertainty estimation techniques to fully exploit the potential of 
Bayesian SegNet. Exploring alternative architectures for optimization is also recommended, as Bayesian SegNet's 
reliance on Monte Carlo sampling for environment perception is computationally intensive. This high resource 
demand could lower the frame rate of visual detection in USVs, potentially hindering real-time processing of 
environmental information.