\begin{abstract}
    This study explores the application of Bayesian SegNet to enhance semantic segmentation for Unmanned Surface 
    Vehicles (USVs), particularly by providing uncertainty estimation to improve robustness in novel environments. 
    Addressing the challenge posed by limited maritime semantic segmentation datasets, the Bayesian SegNet model 
    demonstrated significant improvements over the non-Bayesian baseline, with a 1.3\% increase in Precision 
    ($\mathbf{Pr}$) and a 6.5\% improvement in F1 score ($\mathbf{F1}$). Furthermore, Bayesian SegNet outperformed 
    traditional SegNet in uncertainty estimation, as shown through entropy-based analysis, and exhibited superior 
    generalization capabilities. This was particularly evident in the evaluation of the OASIs dataset, where Bayesian 
    SegNet achieved a 39.77\% higher $\mathbf{F1}$ compared to SegNet, underscoring its ability to make strong 
    inferences in data-scarce conditions. Beyond maritime applications, the methodologies and outcomes of this 
    study hold potential benefits for other USV-related fields where uncertainty estimation and generalization 
    are critical. However, the study acknowledges the need for further refinement in parameter tuning and broader 
    dataset testing to fully leverage the advantages of Bayesian SegNet. Future research should also explore 
    alternative architectures to optimize computational efficiency, given the resource-intensive nature of Monte 
    Carlo sampling required by Bayesian SegNet.
\end{abstract}

\textbf{Keywords:} Semantic segmentation; Deep learning; Bayes methods; Unmanned Surface Vehicles (USVs); 
Uncertainty; SegNet; Obstacle Detection; Convolutional neural networks; MC-dropout; variational inference.